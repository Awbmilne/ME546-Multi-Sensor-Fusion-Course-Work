
\usepackage{graphicx}
\usepackage{pgfplots}
\pgfplotsset{compat=1.18}
\usepackage{caption}
\usepackage{subcaption}
\usepackage{amsmath}
\usepackage{amssymb}
\usepackage{float}
\usepackage{multirow}
\usepackage{multicol}
\def\arraystretch{1.5}
\usepackage{tikz}
\usetikzlibrary{math}
\usepackage{standalone}
\usepackage{fancyvrb}
\usepackage{refcount}
\usepackage{indentfirst} % Indent first line of first paragraphs
\usepackage{parskip} % Configure paragraph spacing
\usepackage{tocloft}
\usepackage{gensymb}
\usepackage{tabularx}
\usepackage{xfrac}
\usepackage{color,soul}
\usepackage{fancyhdr}
\pagestyle{fancy}
\usepackage{longtable}
\usepackage{enumitem}

% Auto indent the section names in the TOC
\renewcommand{\numberline}[1]{#1~}

\usepackage[nottoc,notlof,notlot,numbib]{tocbibind} % Add Bibliography to TOC

\setlength{\parindent}{0.5em} % Indent size for start of paragraphs
\setlength{\parskip}{1em} % Spacing between paragraphs

% Setup references to have hyperlinks
\usepackage{hyperref}
\usepackage{breakurl}
\renewcommand\UrlFont{\rmfamily\small\itshape}
\hypersetup{
        colorlinks,
        citecolor=black,
        filecolor=black,
        linkcolor=black,
        urlcolor=black,
        linktoc=all
}

% List of equations
\newcommand{\listequationsname}{\Large List of Equations}
\newlistof{equations}{equ}{\listequationsname}
\newcommand{\equations}[1]{%
\addcontentsline{equ}{equations}{\protect\numberline{\theequation}#1}\par}
\setlength{\cftequationsindent}{1.5em}% indent of items in List of Equations
\setlength{\cftequationsnumwidth}{2.3em}% Width of equation number in List of Equations

% Add Code listings
\usepackage{listings} 
\usepackage{csvsimple}
\usepackage{tabu}
% This is the color used for Language comments below
\definecolor{MyDarkGreen}{rgb}{0.0,0.4,0.0}
\definecolor{backcolour}{rgb}{204,204,204}
% Rename Listings
\renewcommand{\lstlistingname}{Listing} % Rename listing to code
\renewcommand\lstlistlistingname{List of Code Listings} % Rename list of listing to List of Code
% For faster processing, load python syntax for listings
\lstloadlanguages{Python}%
\lstset{language=Python,                        % Use Python
        frame=tb,                               % Frame on Top and Bottom
        breaklines=true,                        % Allow breaking lines
        basicstyle=%
                \ttfamily                       % Use true type font
                \lst@ifdisplaystyle\scriptsize\fi, % Use scriptsize for Display-Style listings
        keywordstyle=[1]\color{blue}\bfseries,  % Language functions bold and blue
        keywordstyle=[2]\color{purple},         % Language function arguments purple
        keywordstyle=[3]\color{blue}\underbar,  % User functions underlined and blue
        identifierstyle=,                       % Nothing special about identifiers
        % Comments small dark green courier
        commentstyle=\usefont{T1}{pcr}{m}{sl}\color{MyDarkGreen}\scriptsize,
        stringstyle=\color{purple},             % Strings are purple
        showstringspaces=false,                 % Don't put marks in string spaces
        tabsize=5,                              % 5 spaces per tab
        %
        %%% Put standard Language functions not included in the default
        %%% language here
        morekeywords={self},
        %
        %%% Put Language function parameters here
        morekeywords=[2]{},
        %
        %%% Put user defined functions here
        morekeywords=[3]{},
        %
        morecomment=[l][\color{blue}]{...},     % Line continuation (...) like blue comment
        numbers=left,                           % Line numbers on left
        firstnumber=1,                          % Line numbers start with line 1
        numberstyle=\tiny\color{blue},          % Line numbers are blue
        stepnumber=1,                           % Line numbers go in steps of 1
        escapeinside={(*@}{@*)}                 % Escape certain text
}

\newcounter{lstlinereffirst}
\newcounter{lstlinereflast}

\newcommand*\sepline{%
  \begin{center}
    \vspace{-1.75ex}
    \rule[1ex]{.95\textwidth}{.5pt}
    \vspace{-1.75ex}
  \end{center}
}

% Enable Minus Sign Unicode Character?
% \DeclareUnicodeCharacter{2212}{-}
